\chapter{Introduction}\label{chapter:intro} % Un'etichetta per riferirsi al capitolo (es. \ref{cap:intro})

Every year, a significant amount of extraterrestrial material falls into the atmosphere of the Earth,
adding up to a hundred tons per year \cite{binzelMeteoritesEarlySolar2006}.
\begin{figure}[h!]
    \centering
    \includegraphics[width=0.8\textwidth]{figures/NEO_chart.png}
    \caption{This chart shows the cumulative number of known Near-Earth Asteroids (NEAs) versus time.
    Totals are shown for NEAs of all sizes, those larger than 140 m in size, and those larger than 1 km in size. \cite{DiscoveryStatistics}}
    \label{fig:NEO_chart}
\end{figure}

\noindent On a daily basis, we detect meter-sized objects entering the atmosphere, but much larger objects lurk around the Earth:
nearly 1000 objects with a size of 1 km or larger are classified as near-Earth objects (NEOs) \cite{DiscoveryStatistics}, with perihelia of 1.3 AU 
or less.
If any of those NEOs were to impact our planet, we would face civilization-threatening effects, and such events are thought to occur on a
million-year timescale \cite{chengAsteroidImpactDeflection2016}.

On February 15, 2013, a small asteroid impacted Earth in Chelyabinsk, Russia without warning, injuring over a thousand people; this event triggered an unprecedented interest in NEOs and emphasized the importance of discovering potentially dangerous asteroids and deflect them.
On this purpose, the Asteroid Impact and Deflection Assessment (AIDA) mission was programmed by NASA and ESA to be a first demonstration to protect the Earth from a potentially dangerous target asteroid. An "impulsive" approach was preferred in order to achieve immediate effects and as a result, the kinetic impactor technique of impacting an incoming object to deflect it was chosen.

Qui parla velocemente di missione DART e HERA, tempistiche della missione e obiettivo.
Aggiungere stato attuale della missione, parlare delle simulazioni, del buco nella letteratura scientifica che giustifica la tesi e della complessità computazionale.

Presentare la mia idea come una possibile soluzione al problema (modello surrogato sicuramente più veloce).

% \begin{table}[h!]
% \centering
% \caption{Confronto Metriche per Classe (Regressione Logistica)}
% \label{tab:lr_per_class}
% \resizebox{\textwidth}{!}{
% \begin{tabular}{@{}l|ccc|ccc|ccc|ccc@{}}
% \toprule
%  & \multicolumn{3}{c|}{\textbf{st0 (1.0\%)}} & \multicolumn{3}{c|}{\textbf{St1 (21.0\%)}} & \multicolumn{3}{c|}{\textbf{St2 (28.4\%)}} & \multicolumn{3}{c}{\textbf{St3 (49.6\%)}} \\
% \cmidrule(r){2-4} \cmidrule(r){5-7} \cmidrule(r){8-10} \cmidrule(l){11-13}
% \textbf{Modello} & \textbf{P} & \textbf{R} & \textbf{F1} & \textbf{P} & \textbf{R} & \textbf{F1} & \textbf{P} & \textbf{R} & \textbf{F1} & \textbf{P} & \textbf{R} & \textbf{F1} \\
% \midrule
% Standard & 0.00 & 0.00 & 0.00 & 0.24 & 0.10 & 0.14 & 0.67 & 0.04 & 0.07 & 0.52 & 0.94 & 0.67 \\
% Undersampling & 0.06 & 1.00 & 0.12 & 0.33 & 0.38 & 0.35 & 0.28 & 0.35 & 0.31 & 0.57 & 0.27 & 0.37 \\
% Oversampling & 0.00 & 0.00 & 0.00 & 0.31 & 0.52 & 0.39 & 0.35 & 0.11 & 0.16 & 0.62 & 0.39 & 0.48 \\
% SMOTE & 0.00 & 0.00 & 0.00 & 0.31 & 0.48 & 0.37 & 0.33 & 0.12 & 0.18 & 0.63 & 0.39 & 0.48 \\
% \bottomrule
% \end{tabular}
% } 
% \end{table}

% \vspace{1cm} % Aggiunge uno spazio verticale tra le tabelle

% \begin{table}[h!]
% \centering
% \caption{Confronto Metriche Riepilogative (Regressione Logistica)}
% \label{tab:lr_averages}
% \begin{tabular}{@{}l|ccc|ccc|c@{}}
% \toprule
%  & \multicolumn{3}{c|}{\textbf{Macro Average}} & \multicolumn{3}{c|}{\textbf{Weighted Average}} & \multicolumn{1}{c}{\textbf{Overall}} \\
% \cmidrule(r){2-4} \cmidrule(r){5-7} \cmidrule(l){8-8}
% \textbf{Modello} & \textbf{P} & \textbf{R} & \textbf{F1} & \textbf{P} & \textbf{R} & \textbf{F1} & \textbf{Accuracy} \\
% \midrule
% Standard LR (Baseline) & 0.35 & 0.27 & 0.22 & 0.50 & 0.49 & 0.38 & 49.5\% \\
% LR + Undersampling & 0.31 & 0.50 & 0.29 & 0.43 & 0.33 & 0.35 & 33.0\% \\
% LR + Oversampling & 0.32 & 0.26 & 0.26 & 0.47 & 0.34 & 0.37 & 34.0\% \\
% LR + SMOTE & 0.32 & 0.25 & 0.26 & 0.47 & 0.33 & 0.37 & 33.0\% \\
% \bottomrule
% \end{tabular}
% \end{table}

% Contenuto: (Stato dell'arte, motivazioni).

% Analisi: Inizia largo (difesa planetaria, missione DART, importanza di capire il destino degli ejecta) e poi stringi il campo.
% Il "buco" nella letteratura scientifica che giustifica la tesi è la complessità computazionale:
% simulare migliaia di particelle per milioni di anni è insostenibile.

% Obiettivo del capitolo: Il lettore deve finire questo capitolo pensando: 
% "Ok, ho capito il problema (caos, costo computazionale) e ho capito perché serve una soluzione (un modello surrogato veloce ed efficiente)".

