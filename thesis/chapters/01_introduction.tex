\chapter{Introduction}\label{chapter:intro} % Un'etichetta per riferirsi al capitolo (es. \ref{cap:intro})

Every year, a significant amount of extraterrestrial material falls in the atmosphere of the Earth,
adding up to a hundred tons per year \cite{Lauretta2006}.
On a daily basis, we detect meter sized objects entering the atmosphere, but much larger objects lurk around the Earth: nearly 40000 objects
with a size of 1 km or larger are classified as near-Earth objects (NEOs) \cite{jpl_cneos_stats}, with perihelia of 1.3 AU or less.
If any of those NEOs were to impact our planet, we would face civilization-threatening effects, and such events are thought to occur on a
million-year timescale \cite{CHENG201627}.

% Contenuto: (Stato dell'arte, motivazioni).

% Analisi: Inizia largo (difesa planetaria, missione DART, importanza di capire il destino degli ejecta) e poi stringi il campo.
% Il "buco" nella letteratura scientifica che giustifica la tua tesi è la complessità computazionale:
% simulare migliaia di particelle per milioni di anni è insostenibile.

% Obiettivo del capitolo: Il lettore deve finire questo capitolo pensando: 
% "Ok, ho capito il problema (caos, costo computazionale) e ho capito perché serve una soluzione (un modello surrogato veloce ed efficiente)".

