\chapter{Introduction}\label{chapter:intro} % Un'etichetta per riferirsi al capitolo (es. \ref{cap:intro})



% Contenuto: (Stato dell'arte, motivazioni).

% Analisi: Perfetto. Inizia largo (difesa planetaria, missione DART, importanza di capire il destino degli ejecta) e poi stringi il campo.
% Il "buco" nella letteratura scientifica che giustifica la tua tesi è la complessità computazionale:
% simulare migliaia di particelle per milioni di anni è insostenibile.

% Obiettivo del capitolo: Il lettore deve finire questo capitolo pensando: 
% "Ok, ho capito il problema (caos, costo computazionale) e ho capito perché serve una soluzione (un modello surrogato veloce ed efficiente)".

