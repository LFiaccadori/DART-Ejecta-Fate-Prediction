\chapter{Introduction}\label{chapter:intro} % Un'etichetta per riferirsi al capitolo (es. \ref{cap:intro})

Every year, a significant amount of extraterrestrial material falls into the atmosphere of the Earth,
adding up to a hundred tons per year\cite{binzelMeteoritesEarlySolar2006}.
\begin{figure}[h!]
    \centering
    \includegraphics[width=0.8\textwidth]{figures/NEO_chart.png}
    \caption{This chart shows the cumulative number of known Near-Earth Asteroids (NEAs) versus time.
    Totals are shown for NEAs of all sizes, those larger than 140 m in size, and those larger than 1 km in size\cite{DiscoveryStatistics}.}\label{fig:NEO_chart}
\end{figure}

\noindent On a daily basis, we detect meter-sized objects entering the atmosphere, but much larger objects lurk around the Earth:
nearly 1000 objects with a size of 1 km or larger are classified as near-Earth objects (NEOs)\cite{DiscoveryStatistics}, with perihelia of 1.3 AU 
or less.
If any of those NEOs were to impact our planet, we would face civilization-threatening effects, and such events are thought to occur on a
million-year timescale\cite{chengAsteroidImpactDeflection2016}.

On February 15, 2013, a small asteroid impacted Earth in Chelyabinsk, Russia without warning, injuring over a thousand people; this event triggered an unprecedented interest in NEOs and emphasized the importance of discovering potentially dangerous asteroids and deflect them.
On this purpose, the Asteroid Impact and Deflection Assessment (AIDA) collaboration was programmed by NASA and ESA to be a first demonstration to protect the Earth from a potentially dangerous target asteroid. An impulsive approach was preferred in order to achieve immediate effects and as a result, the kinetic impactor technique of impacting an incoming object to deflect it was chosen. The original plan for AIDA was to launch two separate spacecrafts, the Asteroid Impact Mission (AIM) and the Double Asteroid Redirection Test (DART). Their target was selected as the binary asteroid system 65803 Didymos, which consists in a fast-spinning primary named Didymos and a smaller secondary named Dimorphos. Neither of those asteroids represent a threat for the Earth but their characteristics made them the ideal targets for the mission. The AIM orbiter should have been launched in 2020 to reach Didymos and study the composition of both asteroids but was later cancelled and substituted by Hera which will reach the system four years after DART.
On the other hand, DART was launched on November 24, 2021 through a Falcon 9 provided by SpaceX and it became the first plantary defence mission to examine the asteroid deflection capability by kinetic impact.
Before impacting Dimorphos, DART deployed a CubeSat named LICIACube (Light Italian CubeSat for Imaging of Asteroids) which observed the impact from outside and conduced some measurements on the crater and ejecta while transiting close to the system. DART reached and impacted Dimorphos on September 26, 2022, with a relative velocity of 6.15 km/s\cite{nakanoNASAsDoubleAsteroid2022}.
LICIACube observed the ejecta from the DART impact while performing a flyby of Dimorphos about 168 s after the collision. The impact ejecta were further observed by Earth and space-based telescopes, revealing also streams and dust tails. Following ground-based observations revealed that the binary orbit period was reduced by 33.0 $\pm$ 1.0 (3$\sigma$) min\cite{chengMomentumTransferDART2023}.


Aggiungere stato attuale della missione, parlare delle simulazioni, del buco nella letteratura scientifica che giustifica la tesi e della complessità computazionale.

Presentare la mia idea come una possibile soluzione al problema (modello surrogato sicuramente più veloce).

% \begin{table}[h!]
% \centering
% \caption{Summary: Physics-Based Model Performance ($C_J, L_z$)}
% \label{tab:physics_summary}
% % \resizebox deve avvolgere l'intero ambiente tabular
% \resizebox{\textwidth}{!}{%
%     \begin{tabular}{@{}l|cc|cc@{}}
%     \toprule
%     & \multicolumn{2}{c|}{\textbf{Logistic Regression}} & \multicolumn{2}{c}{\textbf{Random Forest}} \\
%     \cmidrule(r){2-3} \cmidrule(l){4-5}
%     \textbf{Balancing Technique} & \textbf{W. Avg F1} & \textbf{F1 (Class 0)} & \textbf{W. Avg F1} & \textbf{F1 (Class 0)} \\
%     \midrule
%     Standard (Imbalanced) & 0.33 & \textbf{0.00} & 0.45 & \textbf{0.00} \\
%     Undersampling & 0.43 & \textbf{0.00} & 0.30 & \textbf{0.00} \\
%     Oversampling & 0.38 & \textbf{0.00} & 0.45 & \textbf{0.00} \\
%     SMOTE & 0.37 & \textbf{0.00} & 0.43 & \textbf{0.00} \\
%     Weighted & 0.39 & \textbf{0.00} & 0.45 & \textbf{0.00} \\
%     \bottomrule
%     \end{tabular}
% } % <-- La parentesi graffa chiude \resizebox
% \end{table}

% \begin{table}[h!]
% \centering
% \caption{Confronto Metriche per Classe (Regressione Logistica)}
% \label{tab:lr_per_class}
% \resizebox{\textwidth}{!}{
% \begin{tabular}{@{}l|ccc|ccc|ccc|ccc@{}}
% \toprule
%  & \multicolumn{3}{c|}{\textbf{st0 (1.0\%)}} & \multicolumn{3}{c|}{\textbf{St1 (21.0\%)}} & \multicolumn{3}{c|}{\textbf{St2 (28.4\%)}} & \multicolumn{3}{c}{\textbf{St3 (49.6\%)}} \\
% \cmidrule(r){2-4} \cmidrule(r){5-7} \cmidrule(r){8-10} \cmidrule(l){11-13}
% \textbf{Modello} & \textbf{P} & \textbf{R} & \textbf{F1} & \textbf{P} & \textbf{R} & \textbf{F1} & \textbf{P} & \textbf{R} & \textbf{F1} & \textbf{P} & \textbf{R} & \textbf{F1} \\
% \midrule
% Standard & 0.00 & 0.00 & 0.00 & 0.24 & 0.10 & 0.14 & 0.67 & 0.04 & 0.07 & 0.52 & 0.94 & 0.67 \\
% Undersampling & 0.06 & 1.00 & 0.12 & 0.33 & 0.38 & 0.35 & 0.28 & 0.35 & 0.31 & 0.57 & 0.27 & 0.37 \\
% Oversampling & 0.00 & 0.00 & 0.00 & 0.31 & 0.52 & 0.39 & 0.35 & 0.11 & 0.16 & 0.62 & 0.39 & 0.48 \\
% SMOTE & 0.00 & 0.00 & 0.00 & 0.31 & 0.48 & 0.37 & 0.33 & 0.12 & 0.18 & 0.63 & 0.39 & 0.48 \\
% \bottomrule
% \end{tabular}
% } 
% \end{table}

% \vspace{1cm} % Aggiunge uno spazio verticale tra le tabelle

% \begin{table}[h!]
% \centering
% \caption{Confronto Metriche Riepilogative (Regressione Logistica)}
% \label{tab:lr_averages}
% \begin{tabular}{@{}l|ccc|ccc|c@{}}
% \toprule
%  & \multicolumn{3}{c|}{\textbf{Macro Average}} & \multicolumn{3}{c|}{\textbf{Weighted Average}} & \multicolumn{1}{c}{\textbf{Overall}} \\
% \cmidrule(r){2-4} \cmidrule(r){5-7} \cmidrule(l){8-8}
% \textbf{Modello} & \textbf{P} & \textbf{R} & \textbf{F1} & \textbf{P} & \textbf{R} & \textbf{F1} & \textbf{Accuracy} \\
% \midrule
% Standard LR (Baseline) & 0.35 & 0.27 & 0.22 & 0.50 & 0.49 & 0.38 & 49.5\% \\
% LR + Undersampling & 0.31 & 0.50 & 0.29 & 0.43 & 0.33 & 0.35 & 33.0\% \\
% LR + Oversampling & 0.32 & 0.26 & 0.26 & 0.47 & 0.34 & 0.37 & 34.0\% \\
% LR + SMOTE & 0.32 & 0.25 & 0.26 & 0.47 & 0.33 & 0.37 & 33.0\% \\
% \bottomrule
% \end{tabular}
% \end{table}

% Contenuto: (Stato dell'arte, motivazioni).

% Analisi: Inizia largo (difesa planetaria, missione DART, importanza di capire il destino degli ejecta) e poi stringi il campo.
% Il "buco" nella letteratura scientifica che giustifica la tesi è la complessità computazionale:
% simulare migliaia di particelle per milioni di anni è insostenibile.

% Obiettivo del capitolo: Il lettore deve finire questo capitolo pensando: 
% "Ok, ho capito il problema (caos, costo computazionale) e ho capito perché serve una soluzione (un modello surrogato veloce ed efficiente)".

