\chapter{Introduction}\label{chapter:intro} % Un'etichetta per riferirsi al capitolo (es. \ref{cap:intro})

Roughly 50 to 100 tons of extraterrestrial material falls into the atmosphere of the Earth every year, with infalls of meter
sized bodies a daily event \cite{Lauretta2006}. However, much larger objects lurk nearby, astronomically speaking: nearly 40000 objects
1 km or larger are classified as near-Earth objects (NEOs) \cite{jpl_cneos_stats}, with perihelia of 1.3 astronomical units or less. Impacts of
1 km objects, which would result in civilization-threatening effects, are thought to occur on roughly million-year time
scales \cite{CHENG201627}.

% Contenuto: (Stato dell'arte, motivazioni).

% Analisi: Inizia largo (difesa planetaria, missione DART, importanza di capire il destino degli ejecta) e poi stringi il campo.
% Il "buco" nella letteratura scientifica che giustifica la tua tesi è la complessità computazionale:
% simulare migliaia di particelle per milioni di anni è insostenibile.

% Obiettivo del capitolo: Il lettore deve finire questo capitolo pensando: 
% "Ok, ho capito il problema (caos, costo computazionale) e ho capito perché serve una soluzione (un modello surrogato veloce ed efficiente)".

