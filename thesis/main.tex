% TEMPLATE TESI MAGISTRALE IN ASTROFISICA
 
% --- CLASSE DEL DOCUMENTO ---
% 'report' è standard per le tesi. 
% '12pt' dimensione font, 'a4paper' per il formato pagina.
% 'oneside' è più semplice per la stesura (puoi cambiarlo in 'twoside' per la stampa finale)
\documentclass[12pt, a4paper, oneside]{report}

% --- LINGUA E CODIFICA ---
\usepackage[utf8]{inputenc}     % Codifica per caratteri speciali (accenti, ecc.)
\usepackage[T1]{fontenc}        % Codifica dei font moderna
\usepackage[english]{babel}     % Imposta la lingua italiana (per "Capitolo", "Indice", ecc.)
\usepackage{csquotes}
% --- GESTIONE BIBLIOGRAFIA (MODERNA) ---
% Usa biblatex + biber, molto più potente di bibtex
\usepackage[
    backend=biber,          % Motore da usare
    style=numeric,          % Stile citazioni (es. [1], [2]). Alternativa: 'authoryear'
    sorting=none            % Ordina la bibliografia per apparizione
]{biblatex}
\addbibresource{bibliography.bib} % Nome del file .bib

% --- PACCHETTI SCIENTIFICI E MATEMATICI ---
\usepackage{amsmath}        % Matematica avanzata
\usepackage{amssymb}        % Simboli matematici
\usepackage{amsfonts}       % Font matematici
\usepackage{siunitx}        % Per unità di misura (es. \SI{10}{\kilo\gram})

% --- GRAFICA E TABELLE ---
\usepackage{graphicx}       % Per includere immagini
\usepackage{booktabs}       % Per tabelle professionali (comandi: \toprule, \midrule, \bottomrule)
\usepackage{caption}        % Per personalizzare le didascalie
\captionsetup{tableposition=top,figureposition=bottom,font=small}

% --- LAYOUT E ALTRO ---
\usepackage{geometry}       % Per impostare i margini
\geometry{a4paper, top=3cm, bottom=3cm, left=3.5cm, right=3cm}
\usepackage{emptypage}      % Non numera le pagine vuote
\usepackage{hyperref}       % Rende i link cliccabili (Indice, citazioni, URL)
\hypersetup{
    colorlinks=true,
    linkcolor=black,        % Colore link interni (es. Indice)
    citecolor=blue,         % Colore citazioni
    urlcolor=cyan           % Colore URL
}

% --- IMPOSTAZIONI GLOBALI ---
% Imposta la cartella principale per le immagini
\graphicspath{ {figure/} }


%%%%%%%%%%%%%%%%%%%%%%%%%%%%%%%%%%%%%%%%%%%%%%%%%%%%%%%%%%%%%%%%%%%%%%%%%%%%%%%
%
% INIZIO DEL DOCUMENTO
%
%%%%%%%%%%%%%%%%%%%%%%%%%%%%%%%%%%%%%%%%%%%%%%%%%%%%%%%%%%%%%%%%%%%%%%%%%%%%%%%
\begin{document}

% --- COPERTINA (FRONTESPIZIO) ---
% Includiamo il file separato che contiene solo la copertina
% Questa pagina non avrà numero
\begin{titlepage}
    \centering % Centra tutto il contenuto

    % --- LOGO UNIVERSITÀ ---
    % Inserisci il logo della tua università. 
    % L'immagine 'logo_universita.png' deve trovarsi nella cartella 'tesi/figure/'
    \includegraphics[width=0.4\textwidth]{figures/logo_universita.png}    
    \vspace{1cm} % Spazio verticale

    % --- INTESTAZIONE ---
    {\Large \textbf{UNIVERSITÀ DEGLI STUDI DI PADOVA}} \\ % Sostituisci con il nome
    \vspace{0.5cm}
    {\large Department of Physics and Astronomy "Galileo Galilei"} \\ % Sostituisci
    \vspace{0.5cm}
    {\large Master Degree in Astrophysics and Cosmology} % Sostituisci

    \vspace{2.5cm} % Spazio ampio prima del titolo

    % --- TITOLO ---
    {\huge \bfseries Predizione del destino orbitale degli ejecta \par}
    {\huge \bfseries dell'impatto DART \par}
    
    \vspace{0.5cm}
    
    {\Large \bfseries Un approccio basato su modelli surrogati \par}
    {\Large \bfseries di Machine Learning \par}

    \vspace{3cm} % Altro spazio

    % --- INFORMAZIONI AUTORE E RELATORI ---
    % \vfill riempie lo spazio verticale rimanente, spingendo il testo in basso
    \vfill 
    
    % Allinea a sinistra il blocco relatore/candidato
    \begin{flushleft} 
    \begin{tabular}{ll}
        \textbf{Thesis Supervisor:}   & Prof. Francesco Marzari \\[1em] % \\[1em] aggiunge spazio
        \textbf{Thesis Co-supervisor:} & Dr. Alessandro Rossi \\[2em]
        \textbf{Candidate:} & Lorenzo Fiaccadori \\
                                   
    \end{tabular}
    \end{flushleft}

    \vspace{1.5cm}
    
    % --- ANNO ACCADEMICO ---
    {\large Academic Year 2025-2026} % Sostituisci

\end{titlepage}


% --- FRONT MATTER (PAGINE INIZIALI) ---
% Iniziamo la numerazione romana per le pagine preliminari
\pagenumbering{roman}

% --- SOMMARIO/ABSTRACT ---
% Puoi creare un file .tex per l'abstract e includerlo qui
% \input{sommario.tex}

% --- RINGRAZIAMENTI (Opzionale) ---
% \input{ringraziamenti.tex}


% --- INDICE GENERALE ---
\tableofcontents
\cleardoublepage

% --- ELENCO FIGURE E TABELLE (Opzionale ma raccomandato) ---
\listoffigures
\cleardoublepage

\listoftables
\cleardoublepage


% --- CORPO PRINCIPALE DELLA TESI ---
% Iniziamo la numerazione araba (1, 2, 3...) da qui
\cleardoublepage
\pagenumbering{arabic}

% Includiamo i capitoli, uno per file
% Questa struttura è FONDAMENTALE per non perdersi
% \input{capitoli/01_introduzione.tex}
% \input{capitoli/02_dati_e_metodi.tex}
% \input{capitoli/03_modello_ml.tex}
% \input{capitoli/04_risultati_e_discussione.tex}
% \input{capitoli/05_conclusioni.tex}


% --- APPENDICI (Opzionale) ---
\appendix % Da qui in poi, i capitoli saranno "Appendice A", "Appendice B", ...
% \input{appendici/A_algoritmi.tex}


% --- BIBLIOGRAFIA ---
\cleardoublepage
%\printbibliography[title={Bibliografia}] % Stampa la bibliografia


\end{document}

